\documentclass[12pt]{article}

\usepackage[russian]{babel}

\newcommand{\MExpect}{\mathsf{M}}

% xelatex
\usepackage{fontspec}
\setmainfont[
  Ligatures=TeX,
  Extension=.otf,
  BoldFont=cmunbx,
  ItalicFont=cmunti,
  BoldItalicFont=cmunbi,
]{cmunrm}
\usepackage{polyglossia}
\setdefaultlanguage{russian}
\setotherlanguage{english}

\usepackage{indentfirst}
\frenchspacing

% Для математики
\usepackage{amssymb,amsmath}
\usepackage{xfrac}
\usepackage{mathtools}
\setcounter{MaxMatrixCols}{20}
\parindent=24pt
\parskip=0pt
\tolerance=2000

% Для настройки размера страницы
\usepackage{geometry}
\geometry{
	a4paper,
	total={170mm,257mm},
	left=20mm,
	top=20mm,
}

% Для вставки графики
\usepackage{graphicx}

% Для таблиц
\usepackage{longtable}
\usepackage{multirow}
\usepackage{tabu}
\usepackage{colortbl}

% Создаем команду, чтобы переносить текст на новую строку внутри таблицы
\newcommand{\tcell}[2][c]{\begin{tabular}[#1]{@{}c@{}}#2\end{tabular}}
\usepackage{makecell}

% Пакет для списков
\usepackage[ampersand]{easylist}

% Для цветных таблиц
\usepackage{tcolorbox}
\tcbuselibrary{skins}
\tcbuselibrary{breakable}

\usepackage{textcomp}

\begin{document}
\begin{center}
\hfill \break
{\Large Министерство науки и высшего образования Российской Федерации}\\
\hfill \break
{\large Федеральное государственное бюджетное образовательное учреждение высшего образования}\\ 
{\normalsize\textbf{<<НОВОСИБИРСКИЙ ГОСУДАРСТВЕННЫЙ ТЕХНИЧЕСКИЙ УНИВЕРСИТЕТ>>}}\\
\hfill \break
\includegraphics{nstu_logo.eps}\\
{\large Кафедра прикладной математики}\\
\hfill \break
\hfill \break
\hfill \break
{\Large\textbf{РГЗ по курсу <<ТВиМС>>}}\\
\hfill \break 
{\Large Часть II}\\
\hfill \break
\hfill \break
\hfill \break
\hfill \break
\hfill \break
\begin{tabu}{ll}
\textbf{Факультет:} & ПМИ \\
\textbf{Группа:} & ПМ-63 \\
\textbf{Студент:} & Шепрут И.И. \\
\textbf{Вариант:} & 22 \\
\textbf{Преподаватели:} & Постовалов С.Н.\\
&Веретельникова И.В. \\
\end{tabu} \\
\hfill \break
\hfill \break
\hfill \break
\hfill \break
\hfill \break
\hfill \break
\hfill \break
{\large Новосибирск\\2018}
\end{center}
\thispagestyle{empty}
\newpage 
\setcounter{page}{1}

\section{Задание \textnumero1 (1.18)}

\subsection{Условие}

В следующей таблице представлены результаты измерений $CO_2$ в граммах на литр в партии газированных напитков.

Требуется проверить гипотезу о согласии полученной выборки с нормальным распределением.

\noindent\begin{tabu}{|X[c, -1]|X[c, -1]|X[c, -1]|X[c, -1]|X[c, -1]|X[c, -1]|X[c, -1]|X[c, -1]|X[c, -1]|X[c, -1]|}
\hline
7.30&7.00&7.20&6.50&7.00&7.00&7.20&7.20&6.80&6.80\\ \hline
6.40&6.80&6.80&6.60&6.90&7.20&6.60&7.30&7.00&6.80\\ \hline
6.70&6.70&6.40&6.80&7.00&6.40&6.80&6.80&7.20&7.20\\ \hline
6.90&7.10&7.40&7.00&7.20&6.80&7.00&7.40&6.60&7.00\\ \hline
6.30&6.60&7.20&6.60&7.20&6.20&7.00&7.20&6.60&6.80\\ \hline
6.50&7.00&6.80&7.00&7.00&6.40&7.20&7.40&7.10&7.00\\ \hline
7.10&7.10&6.90&7.10&6.80&7.40&7.00&6.80&6.60&6.80\\ \hline
\end{tabu}

\subsection{Решение}

Данная гипотеза является сложной. Оценки параметров будем находить с помощью ОМП, а проверять гипотезу при помощи критерия типа Колмогорова (поскольку случайная величина непрерывна).

\subsubsection{Нахождение параметров при помощи ОМП}

Плотность нормального распределения:

\begin{equation}\label{norm_density}
	f(x) = \frac{1}{\sigma\sqrt{2\pi}} e^{-\frac{(x-\mu)^2}{2\sigma^2}}
\end{equation}

Функция максимального правдоподобия:

\begin{equation}\label{true_eq}
	L(\mathbb{X}_n,\ \theta) = \prod_{i=1}^n f(X_i,\ \theta)
\end{equation}

Оценки необходимых параметров находятся из системы уравнений, подставляя \eqref{norm_density} и \eqref{true_eq}:

$$
\frac{\partial \ln L(\mathbb{X}_n, \theta)}{\partial \theta_i} = 0,\ \quad i=1, \dots, r
$$

Где $r = 2$, $\theta_1 = \mu$, $\theta_2 = \sigma$.

Оценки, полученные при помощи ISW:

\begin{equation}\label{marks}
\hat{\mu} \approx 6.9071,\ \hat{\sigma} \approx 0.2866
\end{equation}

\subsubsection{Проверка гипотезы при помощи критерия типа Колмогорова}

Зададим уровень значимости $\alpha = 0.05$.

Статистика этого критерия имеет вид:

$$
S_{K} = \frac{6nD_n+1}{6\sqrt{n}}
$$

$$
D_n = \max\{D_n^+,\ D_n^-\}
$$

$$
D_n^+ = \max_{1\leq i \leq n}\{\sfrac{i}{n}-F(X_{(i)},\ \theta)\}
$$

$$
D_n^- = \max_{1\leq i \leq n}\{F(X_{(i)},\ \theta)-\sfrac{(i-1)}{n}\}
$$

При помощи ISW и полученных оценок \eqref{marks} получаем $S_K \approx 1.2031$, а по таблице получаем $S_{0.05} = 1.3581$, и т. к. $S_K < S_{0.05}$, то \textbf{гипотеза о согласии распределения выборки с нормальным распределением не отвергается.}

Таким образом, количество углекислого газа в газированных напитках подчиняется нормальному распределению.

\section{Задание \textnumero2 (2.15)}

\subsection{Условие}

Рудник за отчетный период выдавал руду из трех экслпуатационных блоков (А1, А2, А3). Горно-геологические условия разработки во всех блоках примерно одинаковы. Идентична организация, технология и механизация работ в блоках. Из каждой вагонетки бралась товарная проба. По данным опробования и химических анализов каждой пробы определено среднее содержание металла в рудах каждой вагонетки. Статистические данны приведены в таблице.

Проверить гипотезу о независимости содержания металла в вагонетке от эксплуатационного блока.

\noindent\begin{tabu}{|X[c, -1]|X[c, -1]|X[c, -1]|X[c, -1]|X[c, -1]|X[c, -1]|}
\hline
\multirow{2}{*}{Выдано за отчетный период} & \multicolumn{4}{c|}{\tcell{Число вагонеток с\\содержанием металла в \%}} & \multirow{2}{*}{Всего} \\ \cline{2-5}
 & 1-3\% & 3-5\% & 5-7\% & 7-9\% & \\ \hline
Из блока А1 & 180 & 80 & 60 & 20 & 340 \\ \hline
Из блока А2 & 90 & 140 & 80 & 20 & 330 \\ \hline
Из блока А3 & 60 & 140 & 80 & 50 & 330 \\ \hline
Общая численность & 330 & 360 & 220 & 90 & 1000 \\ \hline
\end{tabu}

\subsection{Решение}

Зададим уровень значимости $\alpha = 0.05$.

Для проверки гипотезы независимости воспользуемся критерием $\chi^2$ Пирсона. Данные уже группированы. Статистика для этого критерия вычисляется по формуле:

\begin{equation}\label{xi_2}
	X_n^2 = n\sum_{i, j}\left[\frac{\nu_{ij}^2}{\nu_{i\bullet}\nu_{\bullet j}}\right] - n
\end{equation}

% $$
% \begin{aligned}
% X_n^2 &= \frac{1000}{340}\left(
% \frac{180^2}{330}+\frac{80^2}{360}+\frac{60^2}{220}+\frac{20^2}{90}
% \right) + \frac{1000}{330}\left(
% \frac{90^2}{330}+\frac{140^2}{360}+\frac{80^2}{220}+\frac{20^2}{90}
% \right) + \\
% &+ \frac{1000}{330}\left(
% \frac{60^2}{330}+\frac{140^2}{360}+\frac{80^2}{220}+\frac{50^2}{90}
% \right) - 1000 = 
% 1000\left(\frac{677}{1683}+\frac{1114}{3267}+\frac{10}{27}-1\right) \approx
% 113.6138
% \end{aligned}
% $$

Получаем $ X_n^2 \approx 113.6138 $. Число степеней свободы: $(s-1)(k-1)=(3-1)(4-1)=6$, по таблице получаем $S_{0.05} = 12.6$. Поскольку $X_n^2>S_{0.05}$, то \textbf{гипотеза о независимости содержания металла в вагонетке от эксплуатационного блока отвергается.}

Таким образом, содержание металла в вагонетке зависит от номера эксплуатационного блока.


\section{Задание \textnumero3 (3.20)}

\subsection{Условие}

В таблице приведены данные о распределении свинца в пробах, отобранных на двух соседних горизонтах рудника. 

Проверить гипотезу об однородности распределения свинца на разных уровнях рудника.

\noindent{\scriptsize\begin{tabu}{|X[c, -1]|X[c, -1]|X[c, -1]|X[c, -1]|X[c, -1]|X[c, -1]|X[c, -1]|X[c, -1]|X[c, -1]|X[c, -1]|}
\hline
Содержание свинца&0.0-0.1&0.1-0.2&0.2-0.3&0.3-0.4&0.4-0.5&0.5-0.6&0.6-0.7&0.7-0.8&0.8-0.9\\ \hline
270 м.&1&4&8&18&6&7&5&14&2\\ \hline
305 м.&0&4&10&10&12&12&5&13&6\\ \hline
\end{tabu}}

\vspace{1mm}

\noindent{\scriptsize\begin{tabu}{|X[c, -1]|X[c, -1]|X[c, -1]|X[c, -1]|X[c, -1]|X[c, -1]|X[c, -1]|X[c, -1]|X[c, -1]|X[c, -1]|}
\hline
Содержание свинца&0.9-1.0&1.0-1.1&1.1-1.2&1.2-1.3&1.3-1.4&1.4-1.5&1.5-1.6&1.6-1.7&1.7-1.8\\ \hline
270 м.&6&11&9&4&4&2&2&2&1\\ \hline
305 м.&7&7&4&3&1&5&2&0&0\\ \hline
\end{tabu}}

\vspace{1mm}

\noindent{\scriptsize\begin{tabu}{|X[c, -1]|X[c, -1]|X[c, -1]|X[c, -1]|X[c, -1]|X[c, -1]|X[c, -1]|X[c, -1]|X[c, -1]|X[c, -1]|}
\hline
Содержание свинца&1.8-1.9&1.9-2.0&2.0-2.1&2.1-2.2&2.2-2.3&2.3-2.4&2.4-2.5&2.5-2.6&2.6-2.7\\ \hline
270 м.&1&4&2&1&0&0&2&1&2\\ \hline
305 м.&3&1&0&0&0&2&2&2&0\\ \hline
\end{tabu}}

\vspace{1mm}

\noindent{\scriptsize\begin{tabu}{|X[c, -1]|X[c, -1]|X[c, -1]|X[c, -1]|X[c, -1]|X[c, -1]|X[c, -1]|X[c, -1]|X[c, -1]|X[c, -1]|}
\hline
Содержание свинца&2.7-2.8&2.8-2.9&2.9-3.0&3.0-3.1&3.1-3.2&3.2-3.3&3.3-3.4&3.4-3.5&3.5-3.6\\ \hline
270 м.&2&3&1&0&1&1&1&1&2\\ \hline
305 м.&0&1&1&0&0&1&0&0&0\\ \hline
\end{tabu}}

\subsection{Решение}

Для проверки гипотезы независимости воспользуемся критерием $\chi^2$ Пирсона. Статистика для этого критерия высчитывается по формуле \eqref{xi_2}.

Но для начала перегруппируем данные:

\noindent{\scriptsize\begin{tabu}{|X[c, -1]|X[c, -1]|X[c, -1]|X[c, -1]|X[c, -1]|X[c, -1]|X[c, -1]|X[c, -1]|X[c, -1]|X[c, -1]|X[c, -1]|X[c, -1]|X[c, -1]|}
\hline
Содержание свинца&0.0-0.4&0.4-0.8&0.8-1.2&1.2-1.6&1.6-2.0&2.0-2.4&2.4-2.8&2.8-3.2&3.2-3.6\\ \hline
270 м.&31&32&28&12&8&3&7&5&5\\ \hline
305 м.&24&42&24&11&4&2&4&2&1\\ \hline
\end{tabu}}

Зададим уровень значимости $\alpha = 0.05$.

Получаем $X_n^2 \approx 7.75507$. Число степеней свободы: $(s-1)(k-1)=(2-1)(9-1)=8$, по таблице получаем $S_{0.05} = 15.5$. Поскольку $X_n^2<S_{0.05}$, то \textbf{гипотеза о однородности распределений свинца на разных уровнях рудника не отвергается.}

Таким образом, на разных уровнях содержание свинца одинаково распределено.


\end{document}