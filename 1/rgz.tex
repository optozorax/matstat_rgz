\documentclass[12pt]{article}

\usepackage[russian]{babel}

\newcommand{\MExpect}{\mathsf{M}}

% xelatex
\usepackage{fontspec}
\setmainfont[
  Ligatures=TeX,
  Extension=.otf,
  BoldFont=cmunbx,
  ItalicFont=cmunti,
  BoldItalicFont=cmunbi,
]{cmunrm}
\usepackage{polyglossia}
\setdefaultlanguage{russian}
\setotherlanguage{english}

\usepackage{indentfirst}
\frenchspacing

% Для математики
\usepackage{amssymb,amsmath}
\usepackage{xfrac}
\usepackage{mathtools}
\setcounter{MaxMatrixCols}{20}
\parindent=24pt
\parskip=0pt
\tolerance=2000

% Для настройки размера страницы
\usepackage{geometry}
\geometry{
	a4paper,
	total={170mm,257mm},
	left=20mm,
	top=20mm,
}

% Для вставки графики
\usepackage{graphicx}

% Для таблиц
\usepackage{longtable}
\usepackage{multirow}
\usepackage{tabu}
\usepackage{colortbl}

% Создаем команду, чтобы переносить текст на новую строку внутри таблицы
%\newcommand{\tcell}[2][c]{\begin{tabular}[#1]{@{}c@{}}#2\end{tabular}}
\usepackage{makecell}

% Пакет для списков
\usepackage[ampersand]{easylist}

% Для цветных таблиц
\usepackage{tcolorbox}
\tcbuselibrary{skins}
\tcbuselibrary{breakable}

\begin{document}
\begin{center}
\hfill \break
{\Large Министерство образования и науки Российской Федерации}\\
\hfill \break
{\large Федеральное государственное бюджетное образовательное учреждение высшего образования}\\ 
{\normalsize\textbf{<<НОВОСИБИРСКИЙ ГОСУДАРСТВЕННЫЙ ТЕХНИЧЕСКИЙ УНИВЕРСИТЕТ>>}}\\
\hfill \break
\includegraphics{nstu_logo.eps}\\
{\large Кафедра прикладной математики}\\
\hfill \break
\hfill \break
\hfill \break
{\Large\textbf{РГЗ по курсу <<ТВиМС>>}}\\
\hfill \break 
{\Large Часть I}\\
\hfill \break
\hfill \break
\hfill \break
\hfill \break
\hfill \break
\begin{tabu}{ll}
\textbf{Факультет:} & ПМИ \\
\textbf{Группа:} & ПМ-63 \\
\textbf{Студент:} & Шепрут И.И. \\
\textbf{Вариант:} & 22 \\
\textbf{Преподаватели:} & Постовалов С.Н.\\
&Веретельникова И.В. \\
\end{tabu} \\
\hfill \break
\hfill \break
\hfill \break
\hfill \break
\hfill \break
\hfill \break
\hfill \break
\hfill \break
{\large Новосибирск\\2018}
\end{center}
\thispagestyle{empty}
\newpage 
\setcounter{page}{1}

\section{Постановка задачи}

Пусть $X_1, X_2, \dots, X_n$ --- выборка из распределения Вейбулла, с неизвестным параметром $\lambda$ и известным $\alpha$, обладающее плотностью: 
\begin{equation}\label{density}
f(x)=\frac{\alpha x^{\alpha-1}}{\lambda^\alpha} \exp\left\{-\left(\frac{x}{\lambda}\right)^\alpha\right\},\ x \geqslant 0,\ \lambda > 0,\ \alpha > 0
\end{equation}

\begin{easylist}
\ListProperties(Hang1=true)
& Найти точечную оценку неизвестного параметра $\theta$ (или некоторой функции $\tau(\theta)$) по методу моментов или по методу максимального правдоподобия. Проверить полученную оценку на несмещенность, состоятельность и эффективность.
& Найти достаточную статистику.
& Найти функцию $\tau(\theta)$, допускающую эффективную оценку.
& Построить точный доверительный интервал.
& Построить асимптотический доверительный интервал.
\end{easylist}

\section{Решение}

\subsection{Нахождение точечной оценки}

\noindent Вычисляем момент произвольного порядка для \eqref{density}:
\begin{align}
&\MExpect X_i^k&&= \int\limits_0^\infty x^k f(x)\,dx = \int\limits_0^\infty x^k\frac{\alpha x^{\alpha-1}}{\lambda^\alpha} \exp\left\{-\left(\frac{x}{\lambda}\right)^\alpha\right\}\,dx = \frac{\alpha}{\lambda^a}\int\limits_0^\infty x^{\alpha+k-1} \exp\left\{-\left(\frac{x}{\lambda}\right)^\alpha\right\}\,dx = \notag\\
&&&= \left|
\begin{aligned}
	&t=\left(\frac{x}{\lambda}\right)^\alpha\!, x=\lambda t^{\frac{1}{\alpha}}\\
	&a_t = 0, b_t = \infty \notag \\
	&dx=\frac{\lambda}{\alpha}t^{\frac{1}{\alpha}-1}\,dt
\end{aligned}\right| = \frac{\alpha}{\lambda^a}\int\limits_0^\infty \left(\lambda t^{\frac{1}{\alpha}}\right)^{\alpha+k-1}\cdot e^{-t}\cdot\frac{\lambda}{\alpha}t^{\frac{1}{\alpha}-1}\,dt = \lambda^k\int\limits_0^\infty t^{\left(1+\frac{k}{\alpha}\right)-1} e^{-t}\,dt = \notag\\
&&&= \boxed{\lambda^k\Gamma \left(1+\frac{k}{\alpha} \right) = \MExpect X_i^k}
\end{align}

\noindent Вычисляем математическое ожидание для \eqref{density}:

\begin{equation}\label{mx}
\boxed{\MExpect X_i = \lambda\Gamma \left(1+\frac{1}{\alpha} \right)}
\end{equation}

\noindent Вычисляем дисперсию для \eqref{density}:
\begin{equation}\label{dx}
\begin{aligned}
&\Variance X_i &&= \MExpect X_i^2 - \left(\MExpect X_i\right)^2 = \lambda^2\Gamma \left(1+\frac{2}{\alpha} \right) - \lambda^2\Gamma^2 \left(1+\frac{1}{\alpha} \right) = \\
&&&= \boxed{\lambda^2\left[\Gamma \left(1+\frac{2}{\alpha} \right) - \Gamma^2 \left(1+\frac{1}{\alpha} \right)\right] = \Variance X_i}
\end{aligned}
\end{equation}

\subsubsection{Нахождение оценки методом моментов}
\noindent Приравниваем теоретический \eqref{mx} и выборочный момент:
\begin{gather}
	\MExpect X_i = \bar{X} \notag\\
	\hat{\theta}\Gamma \left(1+\frac{1}{\alpha} \right) = \bar{X} \notag\\
	\boxed{\hat{\theta} = \frac{\bar{X}}{\Gamma \left(1+\frac{1}{\alpha} \right)} \label{eval}} \\
	\boxed{T(\mathbb{X}_n) = \frac{\bar{X}}{\Gamma \left(1+\frac{1}{\alpha} \right)}} \label{stat}
\end{gather}

\subsubsection{Проверка на несмещенность}
Статистика $T(\mathbb{X}_n)$ называется \textsl{несмещенной} оценкой параметра $\theta$, если выполняется условие $\MExpect [T(\mathbb{X}_n)] = \theta,\ \forall \theta \in \Theta$.

\noindent Проверяем это для статистики \eqref{stat}:
$$
\begin{aligned}
	&\MExpect [T(\mathbb{X}_n)] &\stackrel{?}{=} \theta \Rightarrow\quad && \MExpect \left[\frac{\bar{X}}{\Gamma \left(1+\frac{1}{\alpha} \right)}\right] &\stackrel{?}{=} \theta \Rightarrow\quad \\[1.3ex]
	\Rightarrow\quad &\frac{\MExpect \bar{X}}{\Gamma \left(1+\frac{1}{\alpha} \right)} &\stackrel{?}{=} \theta \Rightarrow\quad && \frac{\frac{1}{n}\sum_{i=1}^n \MExpect X_i}{\Gamma \left(1+\frac{1}{\alpha} \right)} &\stackrel{?}{=} \theta \Rightarrow\quad \\[1.3ex]
	\Rightarrow\quad &\frac{\frac{1}{n} n \theta \Gamma \left(1+\frac{1}{\alpha} \right)}{\Gamma \left(1+\frac{1}{\alpha} \right)} &\stackrel{?}{=} \theta \Rightarrow\quad && \Aboxed{\theta &= \theta}
\end{aligned}
$$

\begin{center}
\textbf{Таким образом, оценка является несмещенной.}
\end{center}

\subsubsection{Проверка на состоятельность}

\textsl{Критерий состоятельности:} если оценка $\hat{\theta}$ является несмещенной и $\Variance T(\mathbb{X}_n) \xrightarrow{n\to \infty} 0 $, то она также является и состоятельной.

\noindent Проверяем \eqref{eval} на состоятельность:
\begin{equation}\label{dtx}
\Variance T(\mathbb{X}_n) = \Variance \left[\frac{\bar{X}}{\Gamma \left(1+\frac{1}{\alpha} \right)}\right] = \frac{\frac{1}{n} \Variance X_i}{\Gamma^2 \left(1+\frac{1}{\alpha} \right)} = \boxed{\frac{1}{n}\cdot \frac{\theta^2\left[\Gamma \left(1+\frac{2}{\alpha} \right) - \Gamma^2 \left(1+\frac{1}{\alpha} \right)\right]}{\Gamma^2 \left(1+\frac{1}{\alpha} \right)} = \Variance T(\mathbb{X}_n)}
\end{equation}

$$
\Variance T(\mathbb{X}_n) \xrightarrow{n\to \infty} 0
$$

\begin{center}
\textbf{Таким образом, оценка является состоятельной.}
\end{center}

\subsubsection{Проверка на эффективность}
Если распределение является регулярным, и достигается нижняя граница в неравенстве Рао-Крамера \eqref{raokramer}, то оценка называется \textsl{эффективной}.
\begin{equation}\label{raokramer}
\Variance T(\mathbb{X}_n) = \frac{\left[\tau'(\theta)\right]^2}{n\cdot i(\theta)}
\end{equation}

Распределение, очевидно, является регулярным, т. к. плотность \eqref{density} дифференцируема по $\theta$, и множество $\{x : f(x, \theta)=0\}$ не зависит от $\theta$. Поэтому проверим неравенство Рао-Крамера. Но для начала вычислим $i(\theta)$:

\begin{align}
&i(\theta) &&= - \MExpect \left[\frac{\partial^2 \ln f(X, \theta)}{\partial \theta^2}\right] = - \MExpect \left[\frac{\partial^2\left((\alpha-1)\ln x - \alpha\ln\theta -\left(\frac{X}{\theta}\right)^\alpha\right)}{\partial \theta^2}\right] = \MExpect \left[\frac{\partial\left(\frac{\alpha}{\theta} - \frac{\alpha}{\theta}\left(\frac{X}{\theta}\right)^\alpha\right)}{\partial \theta}\right] = \notag\\
&&&= \MExpect \left[-\frac{\alpha}{\theta^2} + \frac{\alpha(\alpha+1)}{\theta^2}\left(\frac{X}{\theta}\right)^\alpha\right] = -\frac{\alpha}{\theta^2} + \frac{\alpha(\alpha+1)}{\theta^{\alpha+2}}\MExpect X^\alpha = -\frac{\alpha}{\theta^2} + \frac{\alpha(\alpha+1)}{\theta^{\alpha+2}} \theta^\alpha\Gamma \left(1+\frac{\alpha}{\alpha} \right) =\notag\\
&&&= \boxed{\frac{\alpha^2}{\theta^2} = i(\theta)}\label{i}
\end{align}

Проверка равенства Рао-Крамера \eqref{raokramer}, используя \eqref{i} и \eqref{dtx}:
$$
\begin{aligned}
\Variance T(\mathbb{X}_n) &\stackrel{?}{=} \frac{\left[\tau'(\theta)\right]^2}{n\cdot i(\theta)} \\
\frac{1}{n}\cdot \frac{\theta^2\left[\Gamma \left(1+\frac{2}{\alpha} \right) - \Gamma^2 \left(1+\frac{1}{\alpha} \right)\right]}{\Gamma^2 \left(1+\frac{1}{\alpha} \right)} &\stackrel{?}{=} \frac{\theta^2}{n\alpha^2} \\
\frac{\Gamma \left(1+\frac{2}{\alpha} \right)}{\Gamma^2 \left(1+\frac{1}{\alpha} \right)} &\stackrel{?}{=} \frac{1}{\alpha^2}+1
\end{aligned}
$$

Это равенство выполняется при $\alpha=1$ и $\alpha \approx -0.6694444982\dotso$ . Первое подходит по условию, а второе нет.

\begin{center}
\textbf{Таким образом, оценка является эффективной только при $\alpha=1$.}
\end{center}

\subsection{Нахождение достаточной статистики}

Воспользуемся \textit{критерием факторизации:} для того, чтобы статистика $T(\mathbb{X}_n)$ была достаточной для модели $F=\{F(x, \theta),\ \theta\in\Theta\}$, необходимо и достаточно, чтобы функция правдоподобия имела вид:
\begin{equation}\label{dostat}
	L(\mathbb{X}_n, \theta) = h(\mathbb{X}_n)\cdot g(T(\mathbb{X}_n), \theta)
\end{equation}

Приводим к необходимому виду:

$$
\begin{aligned}
	L(\mathbb{X}_n, \theta) = \prod_{i=1}^{n} \frac{\alpha X_i^{\alpha-1}}{\theta^\alpha} \exp\left\{-\frac{X_i^\alpha}{\theta^\alpha}\right\} = \underbrace{\alpha^n \theta^{-\alpha n} \exp\left\{-\frac{1}{\theta^\alpha}\overbrace{\sum_{i=1}^n X_i^\alpha}^{T(\mathbb{X}_n)}\right\}}_{g(T(\mathbb{X}_n), \theta)} \cdot \underbrace{\vphantom{
	\alpha^n \theta^{-\alpha n} \exp\left\{-\frac{1}{\theta^\alpha}\overbrace{\sum_{i=1}^n X_i^\alpha}^{T(\mathbb{X}_n)}\right\}
}\prod_{i=1}^n X_i^{\alpha-1}}_{h(\mathbb{X}_n)}
\end{aligned}
$$

\begin{center}
\textbf{Таким образом, статистика $T(\mathbb{X}_n) = \sum_{i=1}^n X_i^\alpha $ является достаточной для модели $F=\{F(x, \theta),\ \theta\in\Theta\}$.}
\end{center}

\subsection{Нахождение функции, допускающей эффективную оценку}

Нам уже известно, что распределение является регулярным, поэтому воспользуемся критерием \textit{эффективности:} $T(\mathbb{X}_n)$ --- эффективная оценка $\tau(\theta)$, если 
$$
	U(\mathbb{X}_n, \theta)\cdot a(\theta) = T(\mathbb{X}_n) - \tau(\theta),
$$
где $a(\theta)$ --- некоторая функция от $\theta$, $U(\mathbb{X}_n, \theta) = \frac{\partial \ln L(\mathbb{X}_n, \theta)}{\partial \theta}$.

\begin{align}
	&U(\mathbb{X}_n, \theta) = \frac{\partial \ln\left[\prod_{i=1}^{n} \frac{\alpha X_i^{\alpha-1}}{\theta^\alpha} \exp\left\{-\frac{X_i^\alpha}{\theta^\alpha}\right\}\right]}{\partial \theta} \notag\\
	&U(\mathbb{X}_n, \theta) = \frac{\partial \left[n\ln\alpha - n\alpha\ln \theta + (\alpha-1)\sum_{i=1}^n \ln X_i - \frac{1}{\theta^\alpha}\sum_{i=1}^n X_i^\alpha\right]}{\partial \theta} \notag\\
	&U(\mathbb{X}_n, \theta) = -\frac{n\alpha}{\theta} + \frac{\alpha}{\theta^{\alpha+1}}\sum_{i=1}^n X_i^\alpha \notag\\
	&\boxed{U(\mathbb{X}_n, \theta) = \frac{n\alpha}{\theta^{\alpha+1}}\left[\frac{1}{n}\sum_{i=1}^n X_i^\alpha - \theta^\alpha \right]}\label{u} \\
	&U(\mathbb{X}_n, \theta) \cdot \underbrace{\vphantom{\frac{1}{n}\sum_{i=1}^n X_i^\alpha} \frac{\theta^{\alpha+1}}{n\alpha}}_{a(\theta)} = \underbrace{\frac{1}{n}\sum_{i=1}^n X_i^\alpha}_{T(\mathbb{X}_n)} - \underbrace{\vphantom{\frac{1}{n}\sum_{i=1}^n X_i^\alpha} \theta^\alpha}_{\tau(\theta)} \notag
\end{align}

\begin{center}
\textbf{Таким образом, статистика $T(\mathbb{X}_n) = \frac{1}{n}\sum_{i=1}^n X_i^\alpha $ является эффективной оценкой функции $\tau(\theta) = \theta^\alpha$.}
\end{center}

\subsection{Построение точного доверительного интервала}

Для построения доверительного интервала используем центральную статистику.

Так как функция распределения:

$$
F(x, \lambda) = \int\limits_{-\infty}^{x} f(t, \lambda)\,dt = \left\{
\begin{aligned}
	&0,\ &&x < 0,\ \\
	&1-e^{-\left(\frac{x}{\lambda}\right)^\alpha},\  && x \geqslant 0,\
\end{aligned}\right.
$$

монотонно убывает по $\lambda$, то можем взять в качестве центральной статистики функцию

$$
	G(\mathbb{X}_n, \theta) = -\sum_{i=1}^n \ln F(X_i, \theta) = -\sum_{i=1}^n \ln \bigg(1-e^{-\left(\frac{X_i}{\theta}\right)^\alpha}\bigg) \succ \Gamma(\beta, n),\ \beta > 0
$$

Тогда границы доверительного интервала $(\tilde{T_1},\ \tilde{T_2})$ определяются при численном решении уравнений $G(\mathbb{X}_n, T_1) = g_1,\ G(\mathbb{X}_n, T_2) = g_2$, где $g_1 < g_2$, $\tilde{T_1} = \min\{T_1, T_2\},\ \tilde{T_2} = \max\{T_1, T_2\}$ и $\Prob\{g1\leqslant G(\mathbb{X}_n, \theta)\leqslant g_2\} = \gamma$, т. е. $g_1,\ g_2$ должны удовлетворять уравнению:

$$
\int\limits_{g_1}^{g_2} \frac{x^{n-1}e^{-x/\beta}}{\Gamma(n) \beta^n}\,dx = \gamma
$$

\subsection{Построение асимптотического доверительного интервала}
Оценки максимального правдоподобия являются асимптотически эффективными и асимптотически нормальными, следовательно сходятся к стандартному нормальному распределению. Следовательно, случайный интервал $
\left(
	\hat\theta - \sfrac{c_\gamma}{\sqrt{n i(\hat\theta)}}, 
	\hat\theta + \sfrac{c_\gamma}{\sqrt{n i(\hat\theta)}}
\right)
$, где $c_\gamma = \Phi^{-1}\left(\frac{\gamma+1}{2}\right)$, является асимптотическим $\gamma$-доверительным интервалом.

Поэтому для начала построим ОМП-оценку параметра $\lambda$. Воспользуемся \eqref{u}:

$$
U(\mathbb{X}_n, \theta) = 0 \quad\Rightarrow\quad
\frac{n\alpha}{\theta^{\alpha+1}}\left[\frac{1}{n}\sum_{i=1}^n X_i^\alpha - \theta^\alpha \right] = 0 \quad\Rightarrow\quad
\boxed{\hat{\theta} = \sqrt[\alpha]{\frac{1}{n}\sum_{i=1}^n X_i^\alpha}}
$$

А также, используя \eqref{i}, получаем: $ i(\hat\theta) = \frac{\alpha^2}{\hat{\theta}^2} $. Строим асимптотический доверительный интервал:

$$
\left(
	\hat\theta - \frac{c_\gamma}{\sqrt{n i(\hat\theta)}}, 
	\hat\theta + \frac{c_\gamma}{\sqrt{n i(\hat\theta)}}
\right)
$$

$$
\left(
	\sqrt[\alpha]{\frac{1}{n}\sum_{i=1}^n X_i^\alpha}\left(1 - \frac{c_\gamma }{\alpha\sqrt{n}}\right),\
	\sqrt[\alpha]{\frac{1}{n}\sum_{i=1}^n X_i^\alpha}\left(1 + \frac{c_\gamma }{\alpha\sqrt{n}}\right)
\right)
$$

\end{document}